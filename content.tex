\section{Regression}
% TODO explain the regression problem we have chosen to solve
As it is the most clinically relevant parameter,
and also the most demanding to measure,
we have chosen to attempt to predict the insulin levels
from other, more easily determined variables
(excluding, of course, the class variable).

\subsection{Linear regression}

\subsection{Artificial Neural Network}
An ANN with a large number of hidden units in several layers was constructed
(200 units in the first hidden layer, 100 in the second),
and normalization layers were placed before every layer but the last.
This maintains activations at a magnitude that assists in training the network.
As an additional consequence of placing a normalization before the first hidden layer,
the input data would also be normalized.

\subsection{Model comparison}


\section{Classification}

This data set contains an obvious binary attribute
with clinical relevance:
does the patient have diabetes or not?
As such, this question is what we will be
focusing our classification efforts on.

\subsection{Decision tree}

\subsection{Logistic regression}

\subsection{$k$-nearest neighbors}

\subsection{Naive Bayes}

\subsection{Artificial Neural Network}

\subsection{Model comparison}

\section{Previous work}

\appendix
\section{Distribution of responsibilities}